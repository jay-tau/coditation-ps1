% Chapter Template

\chapter{References} % Main chapter title

\label{Chapter7} % Change X to a consecutive number; for referencing this chapter elsewhere, use \ref{ChapterX}

\lhead{Chapter 7. \emph{References}} % Change X to a consecutive number; this is for the header on each page - perhaps a shortened title

\begin{itemize}
    \item “Apache Ant - Welcome.” Ant.apache.org, ant.apache.org/.

    \item Atlassian. “Jira Cloud.” Atlassian, 2019, www.atlassian.com/software/jira.

    \item “Bazel.” Bazel, bazel.build/.

    \item “CMake.” Cmake.org, 2018, cmake.org/.

    \item “Dawn - Git at Google.” Dawn.googlesource.com, dawn.googlesource.com/dawn.

    \item Git. “Git.” Git-Scm.com, 2019, git-scm.com/.

    \item GitHub. “GitHub.” GitHub, 2023, github.com/.

    \item Open3D – a Modern Library for 3D Data Processing. www.open3d.org/.

    \item “SPIR - the Industry Open Standard Intermediate Language for Parallel Compute and Graphics.” The Khronos Group, 20 Jan. 2014, www.khronos.org/spir/. Accessed 17 July 2023.

    \item stevewhims. “High-Level Shader Language (HLSL) - Win32 Apps.” Learn.microsoft.com, learn.microsoft.com/en-us/windows/win32/direct3dhlsl/dx-graphics-hlsl.

    \item “WebGPU.” Www.w3.org, www.w3.org/TR/webgpu/.

    \item “WebRTC Home | WebRTC.” Webrtc.org, 2017, webrtc.org/.

    \item “Wgpu-Native.” GitHub, 14 July 2023, github.com/gfx-rs/wgpu-native. Accessed 17 July 2023.

    \item Vcpkg.io, 2023, vcpkg.io/. Accessed 17 July 2023.
\end{itemize}