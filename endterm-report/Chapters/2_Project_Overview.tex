% Chapter Template

\chapter{Project Overview} % Main chapter title

\label{Chapter1} % Change X to a consecutive number; for referencing this chapter elsewhere, use \ref{ChapterX}

\lhead{Chapter 1. \emph{Project Overview}} % Change X to a consecutive number; this is for the header on each page - perhaps a shortened title

%----------------------------------------------------------------------------------------
%	SECTION 1
%----------------------------------------------------------------------------------------

\section{Project Description}

Project Boxxy focuses on developing an open-source sandbox environment specifically designed to handle graphics payloads.
The project encompasses various components, including a WebRTC module for streaming graphics on the web, a native WebGPU module for real-time graphics rendering on the server, DevSync for seamless real-time code editing, and a remote caching system for compiled resources.

The main objective of Project Boxxy is to create an accessible and cost-effective solution for developers and researchers working on graphics-intensive tasks.
The project addresses the challenge of expensive and limited access to high-end GPUs by offering a specialized sandbox environment.
By leveraging the open-source model, Project Boxxy promotes collaboration, innovation, and inclusivity within the graphics development community.

The sandbox environment provided by Project Boxxy serves as a platform for developers and researchers to experiment, create, and deploy graphics-intensive applications without costly hardware investments.
Through the integration of the WebRTC component, graphics streams can be efficiently delivered on the web, enabling remote access and interaction with graphical applications.
The native WebGPU component ensures real-time graphics rendering capabilities on the server, facilitating high-performance graphics processing within the sandbox environment.
Additionally, DevSync allows for real-time code editing, enhancing productivity and teamwork.
The remote caching system optimizes performance by reducing the need for repetitive compilation, resulting in faster execution times for resource-intensive tasks.

Overall, Project Boxxy aims to democratize GPU resources and empower developers and researchers by providing an affordable, accessible, and feature-rich sandbox environment for developing and deploying graphics payloads.
%----------------------------------------------------------------------------------------
%	SECTION 2
%----------------------------------------------------------------------------------------

\section{Motivation Behind the Project}

The motivation behind Project Boxxy stems from the high cost and limited accessibility of high-end GPUs, which pose significant challenges for developers and researchers working on resource-intensive tasks. The expense of acquiring high-end GPUs makes them out of reach for many individuals and organizations, hindering their ability to delve into graphics development and innovation.

To address this issue, Project Boxxy aims to provide an alternative solution by offering an affordable and accessible sandbox environment designed for developing and deploying graphics payloads.
By creating this platform, the project seeks to democratize GPU resources and allow individuals and organizations with limited resources to participate in graphics-driven development actively.

The motivation to offer an open-source sandbox environment for graphics payloads is driven by the belief in collaboration and knowledge sharing within the graphics development community.
By providing a cost-effective solution, Project Boxxy empowers developers and researchers to explore the full potential of graphics-intensive tasks, encouraging innovation and fostering a more inclusive environment.

Ultimately, the motivation behind Project Boxxy is to bridge the gap between the prohibitive costs of high-end GPUs and the need for affordable and accessible GPU resources.
By providing a dedicated sandbox environment, the project enables individuals and organizations to overcome financial barriers and engage in efficient and cost-effective graphics development.